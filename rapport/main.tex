\documentclass{article}
\usepackage[utf8]{inputenc}
\usepackage{geometry}
\usepackage{listings}
\usepackage{xcolor}
\usepackage{hyperref}

% Geometry
\geometry{a4paper, margin=1in}

% Colors for listings
\definecolor{codeblue}{rgb}{0.01, 0.28, 1.0}
\definecolor{codegreen}{rgb}{0,0.6,0}
\definecolor{codegray}{rgb}{0.5,0.5,0.5}
\definecolor{codepurple}{rgb}{0.58,0,0.82}
\definecolor{backcolour}{rgb}{0.95,0.95,0.92}

\lstdefinestyle{mystyle}{
    backgroundcolor=\color{backcolour},
    commentstyle=\color{codegreen},
    keywordstyle=\color{codeblue},
    numberstyle=\tiny\color{codegray},
    stringstyle=\color{codepurple},
    basicstyle=\ttfamily\footnotesize,
    breakatwhitespace=false,
    breaklines=true,
    captionpos=b,
    keepspaces=true,
    numbers=left,
    numbersep=5pt,
    showspaces=false,
    showstringspaces=false,
    showtabs=false,
    tabsize=2
}

\lstset{style=mystyle}

\title{\textbf{Rapport sur les Opérations sur les Fichiers}}
\author{\textbf{Contributions d'Amira}}
\date{\today}

\begin{document}

\maketitle

\section*{Introduction}
Ce rapport détaille les fonctionnalités implémentées par Amira dans le cadre du projet \textbf{Simulateur Simplifié d’un Système de Gestion de Fichiers (SGF)}. La partie d'Amira couvre les \textbf{opérations sur les fichiers}, y compris la recherche, la suppression logique, la suppression physique, la défragmentation, et le renommage des fichiers.

\section*{Fonctionnalités Implémentées}
Amira a implémenté les fonctionnalités suivantes :
\begin{enumerate}
    \item Recherche d'un enregistrement par ID.
    \item Suppression logique d'un enregistrement.
    \item Suppression physique d'un enregistrement.
    \item Défragmentation d'un fichier.
    \item Renommage d'un fichier.
\end{enumerate}

\section*{Pseudocode des Principales Fonctions}

\subsection*{1. Recherche d'un Enregistrement (\texttt{rechercherEnregistrement})}
\textbf{Description :} Recherche un enregistrement par ID et affiche ses détails.

\textbf{Pseudocode :}
\begin{lstlisting}[language=C, caption={Pseudocode pour \texttt{rechercherEnregistrement}}]
Ouvrir le fichier en mode lecture binaire
Si le fichier ne peut pas être ouvert
    Afficher un message d'erreur
Pour chaque enregistrement dans le fichier
    Si l'ID de l'enregistrement correspond à l'ID recherché
        Afficher les détails de l'enregistrement
        Terminer
Afficher un message si l'ID est introuvable
Fermer le fichier
\end{lstlisting}

\subsection*{2. Suppression Logique (\texttt{supprimerEnregistrementLogique})}
\textbf{Description :} Marque un enregistrement comme supprimé en définissant son ID à \texttt{-1}.

\textbf{Pseudocode :}
\begin{lstlisting}[language=C, caption={Pseudocode pour \texttt{supprimerEnregistrementLogique}}]
Ouvrir le fichier en mode lecture/écriture binaire
Si le fichier ne peut pas être ouvert
    Afficher un message d'erreur
Pour chaque enregistrement dans le fichier
    Si l'ID de l'enregistrement correspond à l'ID donné
        Marquer l'enregistrement comme supprimé (ID = -1)
        Sauvegarder les modifications dans le fichier
        Terminer
Afficher un message si l'ID est introuvable
Fermer le fichier
\end{lstlisting}

\subsection*{3. Suppression Physique (\texttt{supprimerEnregistrementPhysique})}
\textbf{Description :} Supprime physiquement un enregistrement en le retirant complètement du fichier.

\textbf{Pseudocode :}
\begin{lstlisting}[language=C, caption={Pseudocode pour \texttt{supprimerEnregistrementPhysique}}]
Ouvrir le fichier original en mode lecture
Créer un fichier temporaire
Pour chaque enregistrement dans le fichier original
    Si l'ID correspond à celui à supprimer
        Ignorer l'enregistrement
    Sinon
        Copier l'enregistrement dans le fichier temporaire
Remplacer le fichier original par le fichier temporaire
Afficher un message de succès
\end{lstlisting}

\subsection*{4. Défragmentation d'un Fichier (\texttt{defragmenterFichier})}
\textbf{Description :} Réorganise un fichier pour supprimer les enregistrements marqués comme supprimés logiquement.

\textbf{Pseudocode :}
\begin{lstlisting}[language=C, caption={Pseudocode pour \texttt{defragmenterFichier}}]
Ouvrir le fichier original en mode lecture
Créer un fichier temporaire
Pour chaque enregistrement dans le fichier original
    Si l'enregistrement n'est pas marqué comme supprimé
        Copier l'enregistrement dans le fichier temporaire
Remplacer le fichier original par le fichier temporaire
Afficher un message de succès
\end{lstlisting}

\subsection*{5. Renommage d'un Fichier (\texttt{renommerFichier})}
\textbf{Description :} Renomme un fichier en changeant son nom.

\textbf{Pseudocode :}
\begin{lstlisting}[language=C, caption={Pseudocode pour \texttt{renommerFichier}}]
Utiliser la fonction rename(ancienNom, nouveauNom)
Si l'opération réussit
    Afficher un message de succès
Sinon
    Afficher un message d'erreur
\end{lstlisting}

\section*{Exemples de Fonctionnement}

\subsection*{Exemple 1 : Recherche d'un Enregistrement}
\textbf{Entrée :}
\begin{itemize}
    \item Nom du fichier : \texttt{test.dat}
    \item ID : \texttt{1}
\end{itemize}
\textbf{Sortie :}
\begin{verbatim}
Enregistrement trouvé : ID: 1, Nom: Amira, Adresse: 16000 Alger
\end{verbatim}

\subsection*{Exemple 2 : Suppression Logique}
\textbf{Entrée :}
\begin{itemize}
    \item Nom du fichier : \texttt{test.dat}
    \item ID : \texttt{2}
\end{itemize}
\textbf{Sortie :}
\begin{verbatim}
Enregistrement avec l'ID 2 supprimé logiquement.
\end{verbatim}

\subsection*{Exemple 3 : Suppression Physique}
\textbf{Entrée :}
\begin{itemize}
    \item Nom du fichier : \texttt{test.dat}
    \item ID : \texttt{3}
\end{itemize}
\textbf{Sortie :}
\begin{verbatim}
Enregistrement avec l'ID 3 supprimé physiquement.
\end{verbatim}

\subsection*{Exemple 4 : Défragmentation}
\textbf{Précondition :} Le fichier \texttt{test.dat} contient 3 enregistrements, dont l'un est marqué comme supprimé (ID = -1).

\textbf{Sortie :}
\begin{verbatim}
Défragmentation terminée. 2 enregistrements restants dans le fichier 'test.dat'.
\end{verbatim}

\subsection*{Exemple 5 : Renommage de Fichier}
\textbf{Entrée :}
\begin{itemize}
    \item Nom actuel : \texttt{test.dat}
    \item Nouveau nom : \texttt{renamed_test.dat}
\end{itemize}
\textbf{Sortie :}
\begin{verbatim}
Le fichier 'test.dat' a été renommé en 'renamed_test.dat'.
\end{verbatim}



\end{document}
