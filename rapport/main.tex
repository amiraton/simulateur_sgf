\documentclass[a4paper, 12pt]{article}
\usepackage[utf8]{inputenc}
\usepackage{amsmath}
\usepackage{amsfonts}
\usepackage{amssymb}
\usepackage{listings}
\usepackage{xcolor}
\usepackage{hyperref}
\usepackage{geometry}
\geometry{margin=1in}

\title{Rapport}
\author{}
\date{}

\begin{document}

\maketitle

\section*{Introduction}
The task involves implementing several operations for managing file records in a simplified file system simulator. These operations include:

\begin{itemize}
    \item Searching for a record by ID.
    \item Logical deletion of a record.
    \item Physical deletion of a record.
    \item File defragmentation to remove logically deleted records.
    \item Renaming a file while updating associated metadata.
\end{itemize}

This report explains the implementation, pseudocode, and examples for each operation.

\section*{Functionality Overview}

\subsection*{1. Search for a Record by ID}
This function searches for a specific record in a file by its unique ID and displays the record's details if found.

\subsubsection*{Pseudocode}
\begin{verbatim}
Open the file in read mode.
For each record in the file:
    If the record ID matches the target ID:
        Display record details.
        Exit.
If no match is found:
    Display "Record not found".
Close the file.
\end{verbatim}

\subsubsection*{Example}
\textbf{Input:} File \texttt{data.dat} containing:
\begin{verbatim}
ID: 1, Name: Alice, Address: Paris
ID: 2, Name: Bob, Address: Berlin
\end{verbatim}
\textbf{Search for:} ID = 2.\newline
\textbf{Output:} "Record found: ID: 2, Name: Bob, Address: Berlin".

\subsection*{2. Logical Deletion of a Record}
This function marks a record as deleted by setting its ID to -1, while keeping the record in the file.

\subsubsection*{Pseudocode}
\begin{verbatim}
Open the file in read-write mode.
For each record in the file:
    If the record ID matches the target ID:
        Set record ID to -1.
        Write the updated record back to the file.
        Exit.
If no match is found:
    Display "Record not found".
Close the file.
\end{verbatim}

\subsubsection*{Example}
\textbf{Input:} File \texttt{data.dat} containing:
\begin{verbatim}
ID: 1, Name: Alice, Address: Paris
ID: 2, Name: Bob, Address: Berlin
\end{verbatim}
\textbf{Delete logically:} ID = 1.\newline
\textbf{Output:} File \texttt{data.dat} updated to:
\begin{verbatim}
ID: -1, Name: Alice, Address: Paris
ID: 2, Name: Bob, Address: Berlin
\end{verbatim}

\subsection*{3. Physical Deletion of a Record}
This function removes a record from the file permanently by rewriting the file without the specified record.

\subsubsection*{Pseudocode}
\begin{verbatim}
Open the file in read mode.
Open a temporary file in write mode.
For each record in the file:
    If the record ID does not match the target ID:
        Write the record to the temporary file.
Replace the original file with the temporary file.
Close both files.
\end{verbatim}

\subsubsection*{Example}
\textbf{Input:} File \texttt{data.dat} containing:
\begin{verbatim}
ID: 1, Name: Alice, Address: Paris
ID: 2, Name: Bob, Address: Berlin
\end{verbatim}
\textbf{Delete physically:} ID = 1.\newline
\textbf{Output:} File \texttt{data.dat} updated to:
\begin{verbatim}
ID: 2, Name: Bob, Address: Berlin
\end{verbatim}

\subsection*{4. File Defragmentation}
This function removes all logically deleted records (marked with ID = -1) and compacts the file.

\subsubsection*{Pseudocode}
\begin{verbatim}
Open the file in read mode.
Open a temporary file in write mode.
For each record in the file:
    If the record ID is not -1:
        Write the record to the temporary file.
Replace the original file with the temporary file.
Close both files.
\end{verbatim}

\subsubsection*{Example}
\textbf{Input:} File \texttt{data.dat} containing:
\begin{verbatim}
ID: -1, Name: Alice, Address: Paris
ID: 2, Name: Bob, Address: Berlin
\end{verbatim}
\textbf{Defragment:}\newline
\textbf{Output:} File \texttt{data.dat} updated to:
\begin{verbatim}
ID: 2, Name: Bob, Address: Berlin
\end{verbatim}

\subsection*{5. File Renaming}
This function renames a file and updates the metadata to reflect the new name.

\subsubsection*{Pseudocode}
\begin{verbatim}
Rename the file using system call.
If successful:
    Update metadata to reflect the new file name.
    Display success message.
Else:
    Display error message.
\end{verbatim}

\subsubsection*{Example}
\textbf{Input:} File name: \texttt{data.dat}.\newline
\textbf{Rename to:} \texttt{records.dat}.\newline
\textbf{Output:} "File 'data.dat' successfully renamed to 'records.dat'."

\section*{Conclusion}
The implementation of these file operations ensures robust functionality with error handling and user feedback. These tasks are critical for managing records in the simplified file system simulator.

\end{document}
